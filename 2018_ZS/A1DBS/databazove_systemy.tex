\documentclass[11pt,a4paper]{book}   % velikost písma, velikost papíru
% PREAMBULE
\usepackage[utf8x]{inputenc}             % formát dokumentu
\usepackage[czech]{babel}               % jazyk dokumentu
\usepackage[T1]{fontenc}                % font písma
\usepackage{listings,graphicx,multicol} % vkládání obrázků
\usepackage[top=2cm,bottom=2cm,left=2cm,right=2cm,marginparwidth=1.75cm]{geometry} % okraje dokumentu
\usepackage{amsmath,amssymb,amsfonts,amsthm}  % matematický formát
\author{Kristýna Tomanová}
\title{Analýza a modelování softwarových systémů}

\begin{document}
Základní pojmy

\textbf{Data} jsou údaje získané pozorováním, měřením nebo zaznamenáním reality - tzn. data jsou nezpracovaná fakta.
Pokud data projdou zpracováním (dostanou strukturu), vznikne z nich \textbf{informace} - smysluplná interpretace dat a vztahů mezi nimi.
Data ukládáme do \textbf{databází} - souboru dat, který slouží pro popis reálného světa. Je to vlastně logicky uspořádáná kolekce navzájem souvisejících dat. Databází je např. evidence knihovny, sklad chemikálií, evidence studentů. 
Aplikační úlohy nad databází řeší \textbf{databázový systém}. Databázový systém obsahuje datové strutkury tak, aby k nim měly optimální přístup všechny úlohy. Řeší uložení, uchování, formátování, zpracování a vyhledávání informací. Databázový systém je složen z databáze, Systému řízení báze dat a databázovou aplikací, dále také hardwarem, správcem a uživateli DBS.

\textbf{Systém Řízení Báze Dat SŘBD} (ang. \textit{DataBase Management System DBSM} je programová vrstva řídící sdílený přístup k databázi. Řesší operace nad databází (definice, vytváření, udržování a řízený přístup), poskytuje všeobecnou možnost dotazování na data. Příkladem DBMS je Oracle, MS SQL Server, MySQL, Sybase, Informix, Progress.
Funkce a služby DBMS
- uložení, vyvolání a aktualizace dat
- uživatelsky přístupný katalog (udržování struktury databáze, uživatelů, aplikacích)
- podpora transakcí (přístup k obsahu databáze nebo jeho změna)
- služby řízení souběžného přístupu
- služby zotavení
- autorizační služby (ochrana před neoprávněným přístupem)
- podpora datové komunikace (integrace se síťovým komunikačním softwarem)
- služby integrity (ochrana kvality dat)
- služby podpory nezávislosti dat
- utility


Historie

\end{document}
