\documentclass[11pt,a4paper]{book}   % velikost písma, velikost papíru
% PREAMBULE
\usepackage[utf8x]{inputenc}             % formát dokumentu
\usepackage[czech]{babel}               % jazyk dokumentu
\usepackage[T1]{fontenc}                % font písma
\usepackage{listings,graphicx,multicol} % vkládání obrázků
\usepackage[top=2cm,bottom=2cm,left=2cm,right=2cm,marginparwidth=1.75cm]{geometry} % okraje dokumentu
\usepackage{amsmath,amssymb,amsfonts,amsthm}  % matematický formát
\author{Kristýna Tomanová}
\title{Analýza a modelování softwarových systémů}

\begin{document}
Základní pojmy

\textbf{Data} jsou údaje získané pozorováním, měřením nebo zaznamenáním reality - tzn. data jsou nezpracovaná fakta.
Pokud data projdou zpracováním (dostanou strukturu), vznikne z nich \textbf{informace} - smysluplná interpretace dat a vztahů mezi nimi.
Data ukládáme do \textbf{databází} - souboru dat, který slouží pro popis reálného světa. Je to vlastně logicky uspořádáná kolekce navzájem souvisejících dat. Databází je např. evidence knihovny, sklad chemikálií, evidence studentů. 
Aplikační úlohy nad databází řeší \textbf{databázový systém}. Databázový systém obsahuje datové strutkury tak, aby k nim měly optimální přístup všechny úlohy. Řeší uložení, uchování, formátování, zpracování a vyhledávání informací. Databázový systém je složen z databáze, Systému řízení báze dat a databázovou aplikací, dále také hardwarem, správcem a uživateli DBS.

\textbf{Systém Řízení Báze Dat SŘBD} (ang. \textit{DataBase Management System DBSM} je programová vrstva řídící sdílený přístup k databázi. Řesší operace nad databází (definice, vytváření, udržování a řízený přístup), poskytuje všeobecnou možnost dotazování na data. Příkladem DBMS je Oracle, MS SQL Server, MySQL, Sybase, Informix, Progress.
Funkce a služby DBMS
- uložení, vyvolání a aktualizace dat
- uživatelsky přístupný katalog (udržování struktury databáze, uživatelů, aplikacích)
- podpora transakcí (přístup k obsahu databáze nebo jeho změna)
- služby řízení souběžného přístupu
- služby zotavení
- autorizační služby (ochrana před neoprávněným přístupem)
- podpora datové komunikace (integrace se síťovým komunikačním softwarem)
- služby integrity (ochrana kvality dat)
- služby podpory nezávislosti dat
- utility


Historie
Od začátku 60. let 19. století se začalo zdokonalovat shromažďování dat. Z děrných štítků na magnetickou pásku až po vynález pevných disků. Používaly se souborově orientované systémy pomocí tzv. \textbf{programově-datové závislosti} - každý program definuje a spravuje vlastní data. Pro databáze se používaly síťové a hierarchické datové modely.
V roce 1970 E. F. Codd definoval \texbf{relační datový model}, který umožnil plošně výkonné transakční zpracování. Od té doby se začaly vyvíjet prototypy relačních DBMS a E-R modely pro návrhy databází.
Od roku 1980 přechází výzkum DBMS na komerční systémy (Oracle). Vznikají paralelní a distribuované databázové systémy a objevují se první objektově orientované databázové systémy.
V 90. letech se rozšířily aplikace pro dolování dat a velké datové sklady, díky rozmachu webového obchodování.
Na přelomu století byly vydány XML a XQuery standardy a multimediální databáze. Od roku 2010 se začalo přecházet na NoSQL.


Databázové systémy
rozdělení podle způsobu ukládání dat a vazeb mezi nimi
- předrelační (souborové, hierarchické, síťové)
- relační (mainframe, PC file-server, klient-server)
- postrelační (objektově orientované, objektově-relační, NoSQL)

  Předrelační databázové systémy
    Souborový model
    - data jsou uložena přímo v souborech počítač
    - soubory jsou nestrukturované (textový formát dat)
    Hierarchický model
    - data jsou organizována do stromové struktury
      - každý syn má jednoho otce
    - při úpravě tabulek je potřeba předefinovat celou databázi
    Síťový model
    - hierarchická struktura
      - každý člen může mít více vlastníků
    - struktura musí být definována předem
    - umožňuje propojení mezi různými typy dat
 Relační databázové systémy
  Relační model
  - založen na matematickém pojmu relačních množin
    - pravidelná struktura dat reprezentována tabulkou
    - vztahy mezi daty se realizují pomocí relační algebry
Postrelační databázové systémy
  Objektově orientovaný model
  - reakce na potřebu ukládání objemných dat (fotografie, grafy, zvuky, videa)
  - základní prvek - objekt
    - uložen ve vlastní databázi
    - datová část (text, grafika, zvuk, videa) a programová část (instrukce pro práci s daty)
  Objektově-relační model
  - relační databáze s objektově orientovaným modelem
  - problémy se stabilitou
  NoSQL model
  - nemají definované databázové schéma
  - při nutném vypořádání se s obrovským množství dat


Architektura databáze

Model ANSI (American National Standards Institute) má za cíl oddělit pohled uživatelů od fyzické reprezentace
  - externí úroveň - uživatelský pohled na databázi
  - konceptuální úroveň - obecný pohled na databázi (data a jejich relace)
  - interní úroveň - fyzická reprezentace databáze
  
  Schéma databáze je celkový popis databáze. Specifikuje se během návrhu a nemění se.
  - externí schéma - může být víc
  - konceptuální schéma - může být jen jedno
  - interní schéma - může být jen jeno
  
 Data v databázi v konkétním časovém okamžiku se nazývají instance databáze.
 
 Nezávislost dat zaručuje, že vyšší úroveň není ovlivněna změnami na nižší úrovni
 - logická nezávislost dat - odolnost externích schémat při změnách konceptuálního
 - fyzická nezávislost dat - odolnost konceptuálního schématu při změnách interního 
\end{document}
