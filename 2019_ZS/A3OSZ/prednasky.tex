Hierarchická struktura počítačového systému
- hardware
- operační systém
- aplikační programy
- uživatelé
Operační systém
- odděluje HW od SW
- spravuje HW
- poskytuje služby SW
- jádro, které běží po celou dobu běhu počítačového systému
    - některé nízkoúrovňové části OS jsou zaváděny na žádost
Pohled shora dolů
  = pohled na OS jako na rozšíření stroje
  - skrývá před programy komplikovanost HW
  - poskytuje jednoduché rozhranní pomocí vysokoúrovňových služeb (= sytémové volání)
  - vytváří virtuální HW (virtual machine)
 Pohled zdola nahoru
  = pohled na OS jako na správce zdrojů
  - působí jako řídící program, který spravuje zdroje a poskytuje je programům (resource allocator/manager)
  - zabraňuje nesprávnému použití počítače
 Vyvolání služby systému
  - režim procesoru
      - režim jádra (privilegovaný)
          - povoleny všechny instrukce
          - pouze vlastní OS (jádro OS)
      - uživatelský
          - některé instrukce jsou zakázány (např. I/O)
          - aplikace a systémové programy
  - postup volání služby aplikací:
      - uložení parametrů (registr, zásobník)
      - vyvolání speciální instrukce
          - vyvolá obslužnou proceduru v jádře
          - přepne do privilegovaného režimu
      - OS zjistí službu, převezme parametry a vyvolá ji
      - návrat do aplikace
      - přepnutí zpět do uživatelského režimu
Základní komponenty OS
- procesy - přidělený výpočetný čas CPU, místo v paměti, vstupní a výstupní soubory
- správa hlavní paměti - alokace a delokace paměti
    - udržuje informaci o využívání paměti
- I/O podsystém
    - správa paměti pro buffering, caching a spooling (Simultaneous Peripherial Operation On Line)
    - vytváření a rušení souborů a adresářů
    - rozvrh diskových operací
    - společné rozhraní ovladačů zařízení
    - ovladače pro specifická zařízení
- ochrana a bezpečnost
    - pouze autorizované procesy
    - specifikace přístupu
    - mechanismus ochrany 
- uživatelské rozhranní
    - součást jádra systému nebo jako samostatný program
    - Command Line Interface CLI - řádkově orientované zadávání příkazů
    - Graphical User Interface GLI - ikony reprezentující programy
- síť
Architektura OS
- Monolitický OS
    - všechny části OS jsou obsaženy v kernelu
    - přímá komunikace částí
    - neomezený přístup kernelu k celému počítačovému systému
    - starší OS (OS/360, VMS, Linux)
- Microkernel
    - poskytuje základní služby
        - nízkoúrovňový paměťový management
        - komunikace mezi procesy
        - základní synchronizace procesů
        - obsluha přerušení
    - další části OS jsou spuštěny mimo jádro s menší úrovní práv
    - OS Mach, GNU Hurd, Win NT
- Vrstevný OS
    - hierarchie procesů
        - nejníže položené vrstvy komunikují s hardwarem
        - každá vyšší vrstva poskytuje abstraktnější virtuální stroj
        - vrstva komunikuje jen se sousední vrsvou
        - možnost výstavby systému od nejnižších vrstev, modularita
        - OS THE
    - funkční hierarchie
        - vznikají závislosti vrstev
- síťové
    - umožňují procesům přistupovat ke zdrojům jiného počítače v síti
    - síťový souborový systém
    - model klient-server
    - komunikace přes síťový protokol
- distribuované 
    - spravuje zdroje na více počítačích
    - mnoho počítačů vytváří jeden superpočítač
    - bez ohledu na umístění zdrojů a procesu
    - těžká implementace
    - Chord, Amoeba, Sprite, Plan 9

