Hierarchická struktura počítačového systému
- hardware
- operační systém
- aplikační programy
- uživatelé
Operační systém
- odděluje HW od SW
- spravuje HW
- poskytuje služby SW
- jádro, které běží po celou dobu běhu počítačového systému
    - některé nízkoúrovňové části OS jsou zaváděny na žádost
Pohled shora dolů
  = pohled na OS jako na rozšíření stroje
  - skrývá před programy komplikovanost HW
  - poskytuje jednoduché rozhranní pomocí vysokoúrovňových služeb (= sytémové volání)
  - vytváří virtuální HW (virtual machine)
 Pohled zdola nahoru
  = pohled na OS jako na správce zdrojů
  - působí jako řídící program, který spravuje zdroje a poskytuje je programům (resource allocator/manager)
  - zabraňuje nesprávnému použití počítače
 Vyvolání služby systému
  - režim procesoru
      - režim jádra (privilegovaný)
          - povoleny všechny instrukce
          - pouze vlastní OS (jádro OS)
      - uživatelský
          - některé instrukce jsou zakázány (např. I/O)
          - aplikace a systémové programy
  - postup volání služby aplikací:
      - uložení parametrů (registr, zásobník)
      - vyvolání speciální instrukce
          - vyvolá obslužnou proceduru v jádře
          - přepne do privilegovaného režimu
      - OS zjistí službu, převezme parametry a vyvolá ji
      - návrat do aplikace
      - přepnutí zpět do uživatelského režimu
Základní komponenty OS
- procesy - přidělený výpočetný čas CPU, místo v paměti, vstupní a výstupní soubory
- správa hlavní paměti - alokace a delokace paměti
    - udržuje informaci o využívání paměti
- I/O podsystém
    - správa paměti pro buffering, caching a spooling (Simultaneous Peripherial Operation On Line)
    - vytváření a rušení souborů a adresářů
    - rozvrh diskových operací
    - společné rozhraní ovladačů zařízení
    - ovladače pro specifická zařízení
- ochrana a bezpečnost
    - pouze autorizované procesy
    - specifikace přístupu
    - mechanismus ochrany 
- uživatelské rozhranní
    - součást jádra systému nebo jako samostatný program
    - Command Line Interface CLI - řádkově orientované zadávání příkazů
    - Graphical User Interface GLI - ikony reprezentující programy
- síť
Architektura OS
- Monolitický OS
    - všechny části OS jsou obsaženy v kernelu
    - přímá komunikace částí
    - neomezený přístup kernelu k celému počítačovému systému
    - starší OS (OS/360, VMS, Linux)
- Microkernel
    - poskytuje základní služby
        - nízkoúrovňový paměťový management
        - komunikace mezi procesy
        - základní synchronizace procesů
        - obsluha přerušení
    - další části OS jsou spuštěny mimo jádro s menší úrovní práv
    - OS Mach, GNU Hurd, Win NT
- Vrstevný OS
    - hierarchie procesů
        - nejníže položené vrstvy komunikují s hardwarem
        - každá vyšší vrstva poskytuje abstraktnější virtuální stroj
        - vrstva komunikuje jen se sousední vrsvou
        - možnost výstavby systému od nejnižších vrstev, modularita
        - OS THE
    - funkční hierarchie
        - vznikají závislosti vrstev
- síťové
    - umožňují procesům přistupovat ke zdrojům jiného počítače v síti
    - síťový souborový systém
    - model klient-server
    - komunikace přes síťový protokol
- distribuované 
    - spravuje zdroje na více počítačích
    - mnoho počítačů vytváří jeden superpočítač
    - bez ohledu na umístění zdrojů a procesu
    - těžká implementace
    - Chord, Amoeba, Sprite, Plan 9
UNIX
- 1969 vytvořen v assembleru
- 1972 přepsán do C
- porting - přenosnost na jiný typ počítače
    projekt GNU - GNU's not UNIX
        - Richard M.Stallman 1984
        - kompletní unixový operační systém založený na svobodném software
        - Free Software Foundation
            - volné studium zdrojového kódu sw
            - volné sdílení sw
            - volné upravování chování programu
            - volné zveřejňování upravené verze softwaru
        - GPL - General Public License
            - zajistit určitá práva vývojářům a uživatelům
            - svoboda spouštět programy
            - svoboda studovat a upravovat zdrojový kód
            - svoboda redistribuovat zdroj a poskytovat úpravy
        - modulární systém
LINUX
- v počátku 1991 byl osobním projektem Linuse Torvaldse
- jednolitý kód s podporou načítání externích modulů
    - modulární monolitické jádro
        - zvýšení stability
        - urychlení běhu jádra
        - zmenšení velikosti samotného jádra
        - zmenšení paměťových nároků
- jádro operačního systému
- kategorie verzí:
    - prepatch / RC
        - zaměřené na vývoj jádra
        - nové funkce, které musí být nejdřív testovány
    - mainline
        - hlavní strom udržovaný Torvaldsem
    - stable
        - vydaný mainline
    - longterm
        - dlouhodobá podpora
- distribuce
    - balík programů, které jsou svázány s jádrem a dalším vybavením systému (knihovny, pomocné nástroje,...)
    - LSB Linux Standard Base
        - sada předpisů astandardů, kterým by měla odpovídat každá distribuce
    - rozdělení
        - binární x zdrojové
        - komerční x nekomerční
        - live distribuce x mini distribuce
    - rozdíly  
        - skladba programů
        - frekvence a způsob vydávání aktualizací
        - instalační program a konfigurační nástroje
        - řešení startovacích skriptů ajejich obsahu
        - organizace adresářů na disku
        - dodatečná úprava programů
        - cena a poskytované služby
    -Slackware GNU/Linux
    - Debian GNU/Linux
        - balíčkovací systém dpkg
    - Fedora Core
        - RedHat
        - RPM pro instalaci a správu balíků
    - Ubuntu
    - Linux Mint
    - SuSE GNU/Linux
    - Gentoo GNU/Linux
    - Linux From Scratch
    - Source Mage GNU/Linux
    - Live Distribuce
        -lze spustit rovnou z média bez nutnosti instalace
        - Slax, Knoppix
- Zavedení systému
    - BIOS - provedení POST
    - spuštění kódu v MBR (GRUB, LILO, SYSLINUX, Loadlin)
        - GRUB GRand Unifiead Bootloader
            - podpora souborového systému při spuštění
            - umožňuje načíst konfigurační soubor z FS
                - změna konfigurace za běhu
            - obsahuje příkazový řádek
            - multibbot = schopnost zřetězení s jiným zavaděčem
            - fáze načítání
                stage 1 - načten z MBR a spuštěn BIOSem
                stage 1.5 - načítání kódem při přístupu na FS
                stage 2 - menu pro výběr OS, možnost úpravy parametrů
    - zavedení jádra Linux
        - přepnutí procesoru do chráněného módu
        - identifikace technického vybavení počítače
        - vytvoření spontánních procesů
    - spuštění startup skriptů a daemonů
        - první spuštěný proces v user space
        - proces init = rodič všech procesů v systému Unix
Fáze jádra
- činnost zavaděče končí předáním řízení kernelu = setup rutina jádra
    - připraví přechod na protected mód
    - dekomprimuje jádro a předá mu řízení
- inicializace registrů
- kontrola typu procesoru
- vysokoúrovňová inicializace
    - datové struktury, systémová konzola, podpora pro dynamické zavádění modulů, VFS (Virtual File System), VM (Virtual Memory manager), vyrovnávací cache, IPC (InterProcess Communication), quota (Subsystém limit a využití disků uživateli), kontroly na chyby hw
- zavedení knihoven (glibc - GNU LIBrary C)
- odstartuje vlákno (thread) pro start procesu init
    - Systém V inicializace (SysV init)
    - zařizuje postupné spouštění skriptů
    - několik úrovní běhu (runlevel 0-6)
Getty
- program, který umožňuje připojit se přes sériové zařízení (virtuální terminál, textový terminál, modem)
- zobrazí přihlašovací prompt - předá uživatelské údaje programu Login
    - ověří heslo a spustí shell
        - kontroluje přihlášení uživatele v /etc/passwd/
            - při podpoře stínování i v /etc/shadow
        - spustí program uvedený v etc/passwd/ (většinou Bash)
- startován v etc/inittab/ procesem init
Bash
- uživatelské rozhraní, čte příkazy a provádí je
- interpret programovacího jazyka
Systemd
- nahrazuje init
- přechází na něj většina distribucí
- zajištění jednotné, centralizované inicializace
- sleduje závislosti procesů a služeb pomocí jejich startování a zastavování
- zná všechny procesy včetně PID a získávání informací o procesech
    - mnohem jednodušší pro systémové administrátory
- podporuje zásobníky - izolovaná prostředí bez virtuálních strojů
    - větší bezpečnost a jednodušší projektování systému
- 
        
    


        
    
            
