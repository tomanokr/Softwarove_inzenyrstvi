% INICIALIZACE
\documentclass[volby]{trida}[RRRR/MM/DD]
%        TRIDA - {article, book, report, letter, slides}
%              ARTICLE - dokument bez kapitol
%        VOLBY
%              - twoside - oboustraný tisk (implicitní pro styl book)
%              - oneside - jednostraný tisk (implicitní pro ostatní styly
%              - openright - kapitoly začínají na pravých stránkách
%              - openany - kapitoly začínají i na levých
%              - a4paper, a5paper, b5paper
%              - letterpaper, legalpaper, executivepaper, landscape
%              - ... ,12pt, 11pt, 10pt, ...
%              - twocolumn (nejde u letter a slides)
%              - leqno - čísla rovnic nalevo místo napravo
%              - fleqn - zarovnání rovnice

% PREAMBULE
\usepackage[volby]{balík}
\usepackage[ utf8 ]{ inputenc }             % formát dokumentu
\usepackage[ czech ]{ babel }               % jazyk dokumentu
\usepackage[ T1 ]{ fontenc }                % font písma
\usepackage{ listings, graphicx, multicol } % vkládání obrázků
\usepackage[ top = 2cm, bottom = 2cm, left = 2cm, right = 2cm, marginparwidth = 1.75cm ]{ geometry } % okraje dokumentu
\usepackage{ amsmath, amssymb, amsfonts, amsthm }  % matematický formát
\usepackage[ colorinlistoftodos ]{ todonotes } % 
\usepackage[ colorlinks = true, allcolors = blue ]{ hyperref }
\usepackage{ mathrsfs }
\usepackage{ wasysym } % smileys
\setlength\parindent{ 0pt } % indent
\author{ Kristýna Tomanová }

% my commands:
\newcommand{\n}{\newline}
\newcommand{\tab}{\hspace{1cm}}

% \newtheorem{definition}{Definition}[section]
\theoremstyle{definition}
\newtheorem{definition}{Definition}
\newtheorem{theorem}{Theorem}
\newtheorem{note}{Note}
\newtheorem{algorithm}{Algorithm}

\newcommand{\baseAll}[3]{\begin{#1}{#2} \begin{flushleft} #3 \end{flushleft} \end{#1}}
\newcommand{\de}[2]{\baseAll{definition}{#1}{#2}}
\newcommand{\ve}[2]{\baseAll{theorem}{#1}{#2}}

\begin{dokument}
.
\end{dokument}
